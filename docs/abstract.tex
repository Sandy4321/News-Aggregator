\titleformat{\section}[block]
{\centering\fontsize{16pt}{18pt}\selectfont\bfseries}{\thesection\cftsecaftersnum}{0.5em}{} % по центру

\section*{Аннотация}
В данной курсовой работе изучаются подходы к сбору, анализу, классификации и кластеризации текстовых данных.
В результате работы удалось собрать корпус новостей, состоящий более чем из 1 млн документов, применить
два подхода к классификации текстовых документов: линейный SVM и градиентный бустинг над деревьями,
а также KMeans и графовые алгоритмы к кластеризации новостей. Для демонстрации практической значимости исследуемых алгоритмов, был разработан 
веб-сервис, который в реальном времени агрегирует новости от различных СМИ, автоматически расставляет для них теги и объединяет в кластеры 
семантически близкие статьи.

\section*{Abstract}
The coursework is dedicated to research of mining, analysis, classification and clustering methods of text data.
As a result, a dataset with more than 1 million Russian news articles was collected. Two approaches for text classification were successfully 
applied: linear SVM and gradient boosted trees. For solving clustering problem KMeans and graph-based algorithms were utilized. In order to show a 
practical  appliance of the research methods, web-service was developed.
The web-service is able to aggregate and analyze text data in the real-time. Moreover, it automatically labels news articles
and cluster them by semantic similarity.

\titleformat{\section}[block]
{\raggedright\fontsize{16pt}{18pt}\selectfont\bfseries}{\thesection\cftsecaftersnum}{0.5em}{} % справа