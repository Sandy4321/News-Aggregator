\titleformat{\section}[block]
{\centering\fontsize{16pt}{18pt}\selectfont\bfseries}{\thesection\cftsecaftersnum}{0.5em}{} % по центру
\section*{Аннотация}
% Больше данных о результатах
% Говно какое-то получилось, завтра перепишу.
В данной работе изучаются алгоритмы классификации и агрегации текста, основанные на
методах и моделях обработки языка, и применяются на русскоязычных новостных статьях.
В процессе работы удалось собрать и обработать большой корпус новостей, на котором
обучены следующие модели: TFIDF, SVM, FastText, word2vec, Kmeans. Классификаторы
показали точность 86-88\%. Для визуализации работы перечисленных алгоритмов на практике
реализован новостной агрегатор в виде web-сервиса, агрегирующий и классифицирующий актуальные
новости с сайтов российских СМИ. 

\section*{Abstract}
The main purposes of this work are the natural language processing models and algorithms of the text aggregation
and classification and application of these models and algorithms on the russian media news articles.
In the course of this work a dataset with the significant number of news was gathered and processed.
On this dataset TFIDF, SVM, FastText, word2vec, Kmeans models were trained. The accuracy of the classifiers
is between 86 and 88 depending on a model. To show how these algorithms are working on practice the news
aggregation web-service was implemented, which aggregates and classifies articles in real-time.
% \lipsum[3-3]

\titleformat{\section}[block]
{\raggedright\fontsize{16pt}{18pt}\selectfont\bfseries}{\thesection\cftsecaftersnum}{0.5em}{} % справа